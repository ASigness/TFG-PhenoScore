\begin{abstract}[catalan]

    La genòmica està revolucionant la medicina de precisió, permetent una avaluació més personalitzada del risc de desenvolupar diverses malalties. En aquest context, els Polygenic Risk Scores (PRS) han emergit com a estratègia clau per a estimar la predisposició genètica a patologies complexes basades en la combinació de múltiples variants genètiques.
    
    No obstant això, la majoria de les eines actuals per a l'anàlisi de PRS presenten barreres significatives per a la seua adopció per part de professionals no especialitzats, ja que requereixen coneixements avançats de bioinformàtica i programació, a més de generar informes tècnics la interpretació dels quals pot resultar desafiadora.
    
    Aquest Treball de Fi de Grau té com a objectiu el disseny i desenvolupament d'una plataforma web accessible i intuïtiva que facilite l'anàlisi de PRS per a un públic més ampli. La plataforma permetrà als usuaris carregar arxius genòmics, configurar l'anàlisi mitjançant models PRS predefinits i visualitzar els resultats de forma clara i comprensible. Es prioritzarà una experiència d'usuari optimitzada, amb informes gràfics i descriptius que simplifiquen la interpretació dels resultats, fomentant la seua aplicació tant en la pràctica clínica com en la investigació.
    
    Mitjançant aquesta solució, es busca eliminar les barreres tecnològiques que actualment limiten l'accés a les anàlisis de PRS, contribuint a la democratització de la genòmica en l'àmbit sanitari. La plataforma facilitarà la integració de la informació genètica en la presa de decisions mèdiques, impulsant l'ús de la medicina de precisió en la prevenció, diagnòstic i tractament de malalties.
    
\end{abstract}
\newpage
\begin{abstract}[spanish]

    La genómica está revolucionando la medicina de precisión, permitiendo una evaluación más personalizada del riesgo de desarrollar diversas enfermedades. En este contexto, los Polygenic Risk Scores (PRS) han emergido como estrategia clave para estimar la predisposición genética a patologías complejas basadas en la combinación de múltiples variantes genéticas. 
    
    No obstante, la mayoría de las herramientas actuales para el análisis de PRS presentan barreras significativas para su adopción por parte de profesionales no especializados, ya que requieren conocimientos avanzados de bioinformática y programación, además de generar reportes técnicos cuya interpretación puede resultar desafiante. 
    
    Este Trabajo de Fin de Grado tiene como objetivo el diseño y desarrollo de una plataforma web accesible e intuitiva que facilite el análisis de PRS para un público más amplio. La plataforma permitirá a los usuarios cargar archivos genómicos, configurar el análisis mediante modelos PRS predefinidos y visualizar los resultados de forma clara y comprensible. Se priorizará una experiencia de usuario optimizada, con reportes gráficos y descriptivos que simplifiquen la interpretación de los resultados, fomentando su aplicación tanto en la práctica clínica como en la investigación. 
    
    Mediante esta solución, se busca eliminar las barreras tecnológicas que actualmente limitan el acceso a los análisis de PRS, contribuyendo a la democratización de la genómica en el ámbito sanitario. La plataforma facilitará la integración de la información genética en la toma de decisiones médicas, impulsando el uso de la medicina de precisión en la prevención, diagnóstico y tratamiento de enfermedades.
    
\end{abstract}
\newpage
\begin{abstract}[english]

    Genomics is revolutionizing precision medicine, allowing for a more personalized assessment of the risk of developing various diseases. In this context, Polygenic Risk Scores (PRS) have emerged as a key strategy to estimate the genetic predisposition to complex pathologies based on the combination of multiple genetic variants.
    
    However, most current tools for the analysis of PRS present significant barriers to their adoption by non-specialized professionals, as they require advanced knowledge of bioinformatics and programming, in addition to generating technical reports whose interpretation can be challenging.
    
    The objective of this Final Degree Project is the design and development of an accessible and intuitive web platform that facilitates the analysis of PRS for a wider audience. The platform will allow users to upload genomic files, configure the analysis using predefined PRS models, and visualize the results in a clear and understandable way. An optimized user experience will be prioritized, with graphical and descriptive reports that simplify the interpretation of the results, promoting its application in both clinical practice and research.
    
    Through this solution, the aim is to eliminate the technological barriers that currently limit access to PRS analysis, contributing to the democratization of genomics in the healthcare field. The platform will facilitate the integration of genetic information into medical decision-making, driving the use of precision medicine in the prevention, diagnosis, and treatment of diseases.
    
\end{abstract}